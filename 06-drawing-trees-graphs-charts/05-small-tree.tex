\documentclass[tikz,border=10pt]{standalone}
\usepackage{hvlogos}
\begin{document}
\begin{tikzpicture}
  \node {\TeX}
    child { node {\LaTeX} }
    child { node {\ConTeXt} }
  ;
\end{tikzpicture}
\end{document}
